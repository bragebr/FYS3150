\section{Conclusion}
The Metropolis-Hastings algorithm proves to be an effective tool for modelling the Ising model in two dimensions. We've managed to produce values for mean energy and magnetization with relative errors in the order of $10^{-5}$ compared to analytical values for the same system. Because of propagation of statistical error, the values gotten for mean heat capacity and susceptibility are not quite as accurate when compared to analytical values. For 20x20 lattices, we see very clearly how the system has an equilibration time, before it reaches a steady state. This fact is confirmed when we consider the amount of accepted states as a function of Monte Carlo cycles. One of the main goals has been to find the critical temperature where a second-order phase transition happens. By modelling increasingly larger lattices, we can see clear indications of the temperature where a phase transition occurs. Using the data from these large lattices, we approximate the critical temperature for an infinitely large lattice. The critical temperature approximated this way differs from the analytical value by only 0.8\%, which is a tremendously good result, and a proof of just how good the Metropolis-Hastings algorithm is when applied to the Ising model. However, as we have encountered frequently during this project, the Metropolis-Hastings algorithm is quite computationally heavy. For lattices of size 100x100, the program had to run for over 20 hours in some cases, even using parallelized code running on 4 CPU cores. For future projects, code should be tested extensively to make sure that it performs faultlessly, and then be run on a more powerful system, like the BigFacet\footnote{\url{https://www.mn.uio.no/fysikk/english/people/adm/almarin/ccse-servers.html}} server. In addition to this, we have also for larger lattices encountered critical slowing down at the critical temperature, a numerical flaw which occurs due to the formation of aligned spin domains, and the corresponding correlation time committed by the single spin flip dynamics.