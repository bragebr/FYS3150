

%%%%%%%%%%%%%%%%%%%%%%%%%%%%%%%%%%%%%%%%%
% Academic Title Page
% LaTeX Template
% Version 2.0 (17/7/17)
%
% This template was downloaded from:
% http://www.LaTeXTemplates.com
%
% Original author:
% WikiBooks (LaTeX - Title Creation) with modifications by:
% Vel (vel@latextemplates.com)
%
% License:
% CC BY-NC-SA 3.0 (http://creativecommons.org/licenses/by-nc-sa/3.0/)
% 
% Instructions for using this template:
% This title page is capable of being compiled as is. This is not useful for 
% including it in another document. To do this, you have two options: 
%
% 1) Copy/paste everything between \begin{document} and \end{document} 
% starting at \begin{titlepage} and paste this into another LaTeX file where you 
% want your title page.
% OR
% 2) Remove everything outside the \begin{titlepage} and \end{titlepage}, rename
% this file and move it to the same directory as the LaTeX file you wish to add it to. 
% Then add \input{./<new filename>.tex} to your LaTeX file where you want your
% title page.
%
%%%%%%%%%%%%%%%%%%%%%%%%%%%%%%%%%%%%%%%%%

%----------------------------------------------------------------------------------------
%	PACKAGES AND OTHER DOCUMENT CONFIGURATIONShttps://www.overleaf.com/project/5f9756abfb48e8000183dcde
%----------------------------------------------------------------------------------------

\documentclass[11pt]{article}

%%Packages

\usepackage{appendix}

%Matematiske
\usepackage{amsmath} %Blant annet referer til likninger
\usepackage{amsfonts} %Inneholder matematiske fonter mathbb


%Grafisk
\usepackage{graphicx}
\usepackage{subcaption}
\usepackage{tikz}
\usetikzlibrary{shapes.geometric, arrows}
\usepackage{float} %Brukes med H for å sette posisjon til figur
\usepackage{pythonhighlight}
\usepackage{listings}
\usepackage{xcolor} % for setting colors
\usepackage{setspace}
\usepackage{multirow}

%Kommentar
\usepackage[colorinlistoftodos]{todonotes}

%Til hyperlinker
\usepackage[colorlinks=true, breaklinks]{hyperref}%Breaklink deler opp linken dersom for lang

\usepackage[utf8]{inputenc} % Required for inputting international characters
\usepackage[T1]{fontenc} % Output font encoding for international characters

\usepackage{baskervillef} % Palatino font

% kode
\usepackage[ruled,vlined]{algorithm2e}
\lstset { %
	language=C++,
	backgroundcolor=\color{black!5}, % set backgroundcolor
	basicstyle=\footnotesize,% basic font setting
}
\usepackage[top=2.5cm, bottom=2.5cm, left=2.5cm, right=2.5cm]{geometry}
\parindent = 0cm


\begin{document}
	%----------------------------------------------------------------------------------------
	%	TITLE PAGE
	%----------------------------------------------------------------------------------------
	
	\begin{titlepage} % Suppresses displaying the page number on the title page and the subsequent page counts as page 1
		\newcommand{\HRule}{\rule{\linewidth}{0.5mm}} % Defines a new command for horizontal lines, change thickness here
		
		\center % Centre everything on the page
		
		%------------------------------------------------
		%	Headings
		%------------------------------------------------
		
		\textsc{\LARGE University of Oslo}\\[1.5cm] % Main heading such as the name of your university/college
		
		\textsc{\Large FYS3150}\\[0.5cm] % Major heading such as course name
		
		\textsc{\large Computational Physics}\\[0.5cm] % Minor heading such as course title
		
		%------------------------------------------------
		%	Title
		%------------------------------------------------
		
		\HRule\\[0.4cm]
		
		{\huge\bfseries The Ising Model for Magnetic Materials\\
		\Large{Modelling Magnetic Phase Transitions Devising the\\ Metropolis - Hastings Algorithm}\\[0.4cm] % Title of your document
		}
		\HRule\\[1.5cm]
		
		%------------------------------------------------
		%	Author(s)
		%------------------------------------------------
		
		%	\begin{minipage}{0.4\textwidth}
		%		\begin{flushleft}
		%			\large
		%			\textit{Author}\\
		%			B.J. \textsc{Blazkowicz} % Your name
		%		\end{flushleft}
		%	\end{minipage}
		%	~
		%	\begin{minipage}{0.4\textwidth}
		%		\begin{flushright}
		%			\large
		%			\textit{Supervisor}\\
		%			Dr. Caroline \textsc{Becker} % Supervisor's name
		%		\end{flushright}
		%	\end{minipage}
		
		% If you don't want a supervisor, uncomment the two lines below and comment the code above
		{\large\textit{Authors}}\\
		\textsc{Lars Kristian Skaar}\\
		\textsc{Håkon Lindholm}\\
		\textsc{Brage Brevig} % Your name
		\\
		\url{https://github.uio.no/lkskaar/FYS3150}
		%------------------------------------------------
		%	Date
		%------------------------------------------------
		
		\vfill\vfill\vfill % Position the date 3/4 down the remaining page
		
		{\large\today} % Date, change the \today to a set date if you want to be precise
		
		%------------------------------------------------
		%	Logo
		%------------------------------------------------
		
		\vfill\vfill
		%\includegraphics[width=0.5\textwidth]{logouio.png}\\[6cm] % Include a department/university logo - this will require the graphicx package
		
		%----------------------------------------------------------------------------------------
		
		%\vfill % Push the date up 1/4 of the remaining page
		
	\end{titlepage}

	%---------------------------------------------------------------------------------%-------
	\newpage
	\setstretch{1.5}
	\setlength\parindent{0pt}
	
	\section{Abstract}
    The Markov Chain Monte Carlo (MCMC) - based Metropolis - Hastings Algorithm (MHA) is here devised to study two - dimensional spin lattices, herein via the Ising model. The study has a ladder - like structure, in which we first study the simple 2x2 spin lattice in order to compare our numerical implementation of the MHA against benchmark analytical solutions. Having studied the performance of our program, we proceed to explore the larger 20x20 spin lattice from which we are able to determine the MCMC equilibration time, which we use to optimize the random sampling of even larger spin lattices. The physical systems, regardless of size, are studied in detail by considering aspects such as energy distribution, expectation values for quantities such as energy and absolute magnetization, and lastly the critical temperature $T_C$. We find that our implementation of the MHA produces numerical results which commit relatives error of order $10^{-5}$ for 10$^{6}$ Monte Carlo sweeps through the 2x2 lattice, and that the larger 20x20 spin lattice reaches a steady state within the range of 5$\cdot10^{3}$ - 10$^{4}$ Monte Carlo sweeps when the spin lattice is initially disordered. The study of even larger systems lead to results which clearly indicate configurational phase transitions, which in turn allowed us to numerically extract the critical temperature in an infinitely large spin lattice. We found this to hold the value of $2.289$, deviating from the theoretical value obtained by Onsager ($T_C = 2/\sqrt{1 + \sqrt{2}} \approx 2.269)$ by 0.8\%. 
	
	
	
	\newpage
	\tableofcontents
	\newpage
	
	\section{Introduction}
  
    
    Setting out, this numerical project has a three-fold goal; studying three differential equation solvers, designing and devising an object oriented code and lastly making a model for our Solar System. We choose to look at this project as an exercise in object orientation, as this is an important programming model which allows developers to design software in which large data bases may be treated seamlessly and with high precision - when a certain data type, or \textit{object} is known, its inherent properties and behaviour may be cast into a class of objects displaying similar properties or behaviour. This allows the same logical structures to be devised across different, yet similar objects. In the first part of this article, we shall more formally define the problem at hand and how we object orient our code and implement the numerical schemes for solving $N$ - body motion in a system of celestial bodies. In this part we also explain how we have performed surveys of our model using logical test functions in lack of analytical solutions we may compare our model against. We later go on to present the myriad of results we have procured in this project and discuss these in light of the goals of the study. We will do this in an orderly fashion. Lastly we make some conclusive remarks on our findings and shed light on prospective changes.  

	\section{Theory \& Background}
\subsection{The SIRS Model}

\label{section: SIRStheory}
The SIRS model is a compartmental model describing the distribution of agents existing in each compartment

\begin{itemize}
    \item Susceptible (S) - agents who are at risk of being infected by the disease,
    \item Infected (I) - agents who are infected by the disease and may pass on the disease to susceptible agents,
    \item Recovered (R) - agents who have recovered from the disease.
    
\end{itemize}

at a given time instance. It is a derivative of the simpler SIR model \cite{SIR}, including here the possibility of transitioning from the state of being 'removed' (recovered or deceased) into a state of being susceptible to future contagion. 
\begin{figure}[H]
    \centering
    \includegraphics{Figures/SIR.PNG}
    \caption{A modular flow chart describing how agents move between the three compartments $S,\ I\ \text{and}\ R$.}
    \label{fig:SIRS}
\end{figure}
An agent may only transition between the network's compartments in a cyclic fashion as suggested by the model's name $S \overset{\alpha}{\to} I \overset{\beta}{\to} R \overset{\gamma}{\to} S$ with respective transition probabilities $\alpha, \ \beta\ \text{and} \ \gamma$. The transmission rate $\alpha$, recovery rate $\beta$ and immunity loss rate $\gamma$ all have units of inverse time. As an example, setting $\alpha = \beta = \gamma = 1$ / day implies that the agents in the network on average contacts, recovers from and loses immunity to the disease within one day and may be taken as a measure of how long the agent exists in a given compartment. In the deterministic picture, we view the occupation of each compartment as a temporally continuous variable such that the dynamics of compartmental transition is governed by the following set of ordinary \textit{non-linear}\footnote{This implicates sensitivity to initial conditions.} differential equations

\begin{align}
    \begin{split}
        S'(t) &= \gamma R(t) - \alpha S(t)I(t)/N\\
        I'(t) &= \alpha S(t)I(t)/N - \beta I(t)\\
        R'(t) &= \beta I(t) - \gamma R(t)
        \label{SIRS}
    \end{split}
\end{align}

where $N = S(t) + I(t) + R(t)$ is the total number of agents in the network, and is for our simplest simulations assumed to be constant as we here omit vital dynamics considerations such as birth rate and natural death rate, assuming then that disease evolves in the network on a time scale which is significantly shorter than the time scale spanning the average agent's lifetime. In the stochastic picture, however, we first observe that for a sufficiently small temporal step $\Delta t$, at most one agent moves from a given compartment to another. As 
$$
\begin{cases}
\text{max}\{\underbrace{\alpha SI\Delta t/N}_{S\to I}\} = \alpha N\Delta t/4\\

\text{max}\{\underbrace{\beta I\Delta t}_{I\to R}\} = \beta N\Delta t \\

\text{max}\{\underbrace{\gamma R \Delta t}_{R\to S}\} = \gamma N\Delta t
\end{cases}
$$
we may require that $\Delta t = \text{min}\{4/\alpha N, 1/\beta N, 1/\gamma N\}$ and reinterpret
$$
\begin{cases}
P(S\to I) = \alpha SI\Delta t/N\\\
P(I\to R) = \beta I\Delta t\\
P(R\to S) = \gamma R \Delta t
\end{cases}
$$

as transition probabilities. In this way, we may implement the SIRS model stochastically by generating (pseudo)random numbers and at each time step allow agents to move in and out of each compartment by proposing transitions and either accepting or rejecting them by measuring each transition probability against the random numbers. It should be duly noted that the SIRS model further works on the basis of the following assumptions\\

\begin{itemize}
    \item The population mixes homogeneously, id est every agent in the network is at all times equally likely to contact, recover and lose immunity to the disease.
    \item Following the homogeneity of the population, all the parameters describing transitions are averages.
    \item The disease has no pertaining incubation time. It is instantly transmitted, and those who contact the disease are immediately moved into compartment $I$.
    \item In the case of adding vital dynamics to the model, we assume that all newborns are initially susceptible.
\end{itemize}

\subsection{Improving the Model}\label{improvingmodel}
In this section we shall explore in what ways we may incorporate different realizations of the SIRS model in order to better capture the true nature of how infectious diseases may spread within a network. The layers of realism we add to the model in this project do not confer the possibility of making real - world predictions for the current, or any other, pandemic as this is a system with very complex behaviour in both temporal and spatial dimensions. Rather, it allows us to heuristically study their impact on the model.\\

In a first attempt of improving the model, we add the realization of vital dynamics across all compartments $S, \ I \ \text{and}\ R$. In many cases, the stages of  identification, outbreak and eradication of an infectious disease may all occur within a time span significantly shorter than the average lifetime of the agents within the network. However, in a few select cases, this is far from the truth. A few examples include the Ugandan trypanosomiasis epidemic (1900 - 1920)\cite{uganda}, the HIV/AIDS pandemic (1981 - present) and the Antonine Plague (165 - 180, Roman Empire)\cite{historypandemics}. Such long lasting diseases not only claim the lives of a significant percentage of the population, but their dynamics are also dependent on the in - and outflow of agents within the network. An improved modular scheme may be visualized as
\begin{figure}[H]
    \centering
    \includegraphics{Figures/SIRVITAL.PNG}
    \caption{A modular flow chart describing the motion of agents between the compartments $S, \ I\ \text{and}\ R$, now adding vital dynamics.}
    \label{SIRVITALCHART}
\end{figure}
where $\mu, \ \mu^\star \ \text{and} \ \epsilon$ denote the average death rate due to natural causes, the death rate due to the infectious disease and the average birth rate respectively. In the deterministic picture, we may construct a new set of differential equations
\begin{align}
    \begin{split}
        S'(t) &= \gamma R(t) - \alpha S(t)I(t)/N(t) - \mu S(t) + \epsilon N(t)\\
        I'(t) &= \alpha S(t)I(t)/N(t) - \beta I(t) - (\mu + \mu^\star)I(t)\\
        R'(t) &= \beta I(t) - \gamma R(t) - \mu R(t)
        \label{SIRS}
    \end{split}
\end{align}

where we now let the total number of agents in the network $N(t)$ be a continuous variable in time. For the stochastic simulation, we determine the temporal resolution $\Delta t$ in the same way as before, now adding the possible transitions $S\to D,\ I\to D \ \text{and} \ R \to D$ for a new compartment $D$ in which deceased agents are moved to, $I \to D_I$ for a new compartment $D_I$ in which agents who die from the disease are moved to, and $B \to S$ for a new compartment $B$ in which newborns are moved from. As before, we may then reconstruct new transition probabilities
$$
\begin{cases}
P(S\to D) = \mu S\Delta t\\
P(I\to D) = \mu I\Delta t\\
P(R\to D) = \mu R \Delta t\\
P(I\to D_I) = \mu^\star I \Delta t\\
P(B\to S) = \epsilon N \Delta t
\end{cases}
$$

As before, the vital dynamics realization will not allow us to make predictions of real - world statistical value, and now mainly because the provided examples of long lasting diseases are very complex in their transmission. HIV/AIDS is vertically transmitted, meaning that offspring are born with the disease, and the two other confer vector transmission, meaning that the disease is transmitted from animal carriers onto agents. Although we do not attempt to implement these conditions in this project, the additional layer of vital dynamics allow us to study the effect of death and birth in combination with the disease.\\

Many infectious diseases, after an initial outbreak, may after some time be categorized as an endemic disease. In this stage, the disease has strongly manifested at an non-zero equilibrium, around which the number of currently infected agents may fluctuate. Such fluctuations are largely conferred by seasonally variable transmission rates, and is the case for diseases like the 'flu' or measles. The variations themselves are to this day studied in detail as they are not fully understood, although the commonly accepted explanations are characteristic pathogen survival outside of host, and the behaviour of susceptible hosts\cite{seasonal1}\cite{seasonal2}. Pathogens like the rota - and noroviruses, causing the 'flu', survive in cold climates unlike many other pathogens, and therefore display annual peaks in infection rates during the colder months of the year. Similarly, measles are a common disease among children, and analysis shows that there is a strong correlation between the rate of infection and the conglomeration of children at school, after - school  activities and the alike, pertaining seasonal variation in transmission rates to the behavioural pattern of the pathogen's host. To model this realization in our program, we define the transmission rate as a function of time by 
\begin{equation}
    \alpha(t) = A\cos(\omega t) + \alpha_0
\end{equation}
in which $\alpha_0$ is the average transmission rate, $A$ is the amplitude of the variation, being the largest deviation from the average transmission rate, and $\omega$ denotes the frequency at which the transmission rate changes. A flow chart describing the dynamics now adding seasonal variation may be taken as Figure \ref{SIRVITALCHART}, substituting $\alpha$ for $\alpha(t)$.\\

In a last attempt to make our model more realistic, we study the effects of vaccination, too. Vaccination is a powerful tool, and in many cases absolutely necessary in order to eradicate a disease in a network. This works by breaking the cyclic nature of the SIRS model, as agents in the susceptible state may be vaccinated and move directly into the recovered state. Agents in the recovered state may be vaccinated such that their possibility of moving into the state of being susceptible becomes zero.
\begin{figure}[H]
    \centering
    \includegraphics{Figures/sirvax.PNG}
    \caption{A final, compound modular flow chart describing all realizations of the SIRS model we have implemented.}
    \label{fig:sirsvax}
\end{figure}
Figure \ref{fig:sirsvax} describes the motion of agents in and out of the three different compartments, now having added the possibility of being vaccinated. This gives rise to $\gamma^\star$, still describing the rate of loss of immunity to the disease, but now only being applicable to agents occupying $R$ whilst not having been vaccinated.\\

For the deterministic approach, a new set of coupled differential equations emerges as

\begin{align}
    \begin{split}
        S'(t) &= \gamma R(t) - \alpha S(t)I(t)/N - f(t)\\
        I'(t) &= \alpha S(t)I(t)/N - \beta I(t)\\
        R'(t) &= \beta I(t) - \gamma R(t) + f(t)
        \label{SIRSvax}
    \end{split}
\end{align}
in which $f(t)$ describes how the rate of vaccination varies with time. It may also be taken as a constant rate. For the stochastic approach, the only change needed in our program is the added transition probability $P(S\to R) = f(t)\Delta t$. 

In this project we study two possible vaccination programs, one in which the agents in the network are continuously vaccinated by $f(t) = \exp(\delta t)$ for some constant $\delta$ as to realize the continuous development of vaccines and their exponentially growing availability. This is of course a highly idealized scenario as this is economically infeasible and rarely occurs. In the second attempt to realize a more common vaccination strategy, we add a rate of vaccination following
\begin{equation}
f(t) = 
    \begin{cases}
    \eta t, \ \ \text{if} \ \cos(t) > 0\\
    0, \ \ \text{else}
    \end{cases}
    \label{fig:pulsevac}
\end{equation}
where $\eta \in \mathbb{R}$. This implementation is meant to simulate pulse vaccination, in which select groups within a network (children, elderly) are vaccinated at regular intervals as a mean of eradicating a disease. As our implementation of the SIRS model does not differentiate between agents (homogeneous mixing), we may still choose to vaccinate a given percentage of the occupants of the Susceptible compartment. Although pulse vaccination programs, such as the Pulse Polio Programme in India are well established, the mathematical and numerical aspects are still in their infancy which is why we find it necessary to study the effects of such a realization, too. Even though the pulse functions devised in other numerical and/or mathematical research papers are more formalized than the one we have chosen to implement, equation \ref{fig:pulsevac} severs its purpose as a mathematical pulse, and is in addition very easy to reformulate\footnote{As an example, we studied regular non-increasing pulses, for which equation \ref{fig:pulsevac} returns a constant.}.\\
\newpage
\subsection{Initial Conditions, Equilibrium and Stability}
In this project we study four different networks

\begin{table}[H]
    \centering
    \begin{tabular}{l|l|l|l|l}
         & A & B & C & D\\
         \hline
         $\alpha$ & 4.0 & 4.0 & 4.0 & 4.0\\
         $\beta$ & 1.0 & 2.0 & 3.0 & 4.0\\
         $\gamma$ &0.5 &0.5 &0.5 &0.5\\
    \end{tabular}
    \caption{The transition rates for the four different networks we study.}
    \label{tab:initconditions}
\end{table}

In all four networks, each compartment is initialized as $S_0 = 300$, $I_0 = 100$ and $R_0 = 0$. Upon adding different layers of realistic improvements to our model, we have devised many different initial conditions, all of which will be specified in the presented results.\\

The set of equations \eqref{SIRS} define a conservative system as the population remains constant under the assumption that we may omit vital dynamics in the first place. By that, we eventually reach an equilibrium state $(S_\infty, I_\infty, R_\infty)$ for which it is possible to derive analytical values for the fraction of occupants in each compartment by setting each derivative equal to zero. We rather choose to give the total number of occupants in each compartment
at equilibrium by

\begin{align}
    \begin{split}
        S_\infty &= N\frac{\beta}{\alpha}\\
        I_\infty &= \frac{N(1-\frac{\beta}{\alpha})}{1+\beta\gamma}\\
        R_\infty &= \frac{N\beta}{\gamma}\frac{1-\frac{\beta}{\alpha}}{1+\frac{\beta}{\gamma}}
        \label{sirseq}
    \end{split}
\end{align}

Notice in \eqref{sirseq} that for any values of $\alpha$ and $\beta$ such that $\beta < \alpha$, the number of infected agents at equilibrium is non - zero. In this project we only study the effect of varying the rate of recovery $\beta$ in the simple SIRS model, mainly because this variable is more easily controllable by real - world health measures. It should be noted however, that the rate of transmission, or contact rate, is to some extent also controllable by imposing social measures such as wearing masks, increasing the availability of disinfectants or city-wide lockdown. We have not studied this parameter in detail as of now, as we rather would control this variable in a model which takes heterogeneous mixing into account, and is therefore a prospect for our future selves.    
	\section{Numerical Methods and Implementation}
	\subsection{Implementation and Class Hierarchy}
	
	As was mentioned in the introduction of the article, the goal of this numerical project is not only to study differential solvers and the solar system, but also an exercise in designing object oriented code. We therefore see it as necessary to briefly explain the class hierarchy of our program, and how this implementation simultaneously is the foundation for an object oriented, 'recyclable' code which may be extended to study systems beyond our Solar System.\\
	
	Our program is based on three abstract classes; \texttt{System}, \texttt{Planet} and \texttt{Solver}. The class \texttt{System} contains settings needed in place to add objects to a list. Said objects are then contained in the system we wish to study. The more interesting class \texttt{Planet} constructs and declares the planets - or objects - we wish to add to our system. Each planet has three fundamental attributes - position, velocity and mass - all which are defined in this class. This class illustrates why our program could be cast into different physical systems. Given that a system contains objects of similar attributes as the planets we are studying in this project, they would be easily defined numerically in \texttt{Planet} - a class which by all means could be called \texttt{Particles} or \texttt{Objects}. It should be noted, however, that Newton's law of gravitation is implemented in \texttt{Planet} as the force acting on a planet from all other planets in the same system. This would naturally have to be changed given that a different physical system is under the influence of a different (or several other) forces when re-devising the program. Lastly, the class \texttt{Solver} is where the magic happens. In this class, the \texttt{EulerForward()}, \texttt{EulerCromer()} and \texttt{VelocityVerlet()} - solvers are all declared. This class inherits objects from \texttt{Planet} and systems of said objects from \texttt{System}, allowing us to perform operations on them to produce results.
	\subsection{A Psuedocode for Calculating the Net Force on a Given Planet}
	Going forward, we shall shed light on the numerical implementation of the three algorithms we are studying. In each of them, the calculation of the net force acting on a given planet in a given system acts as the backbone of the algorithm, which is why we introduce it here. We will also present the implementation of each algorithm as psuedocode, which is why we encourage the reader to visit our GITHUB repository linked on the first page in the article to view all code. 
	
	\IncMargin{1em}
	\begin{algorithm}[H]
		\caption{Calculate Net Force}
		\label{algonetforce}
		\SetAlgoLined
		
		\SetKwData{Left}{left}\SetKwData{This}{this}\SetKwData{Up}{up}
		\SetKwFunction{Union}{Union}\SetKwFunction{FindCompress}{FindCompress}
		\SetKwInOut{Input}{input}\SetKwInOut{Output}{output}
		\Input{System}
		\Output{Net force on Planet ($\mathbf{F}\in \mathbb{R}^3$)}
		\textbf{Var}:\\
		i, n, planet\_idx, otherPlanet\_idx, System.TotalPlanets $\in \mathbb{N}$\\
		System.AllPlanets $\in \mathbb{N}^{1\times P}, \ \ \ P = \text{Number of planets in System}$
		
		\For{i \KwTo $n-1$}{
		\For{\text{Planet\_idx} \KwTo \text{System.TotalPlanets-1}}{current\_planet = System.AllPlanets[planet\_idx]\\
		
		\For{otherPlanet\_idx \KwTo System.TotalPlanets - 1}{current\_planet.Force += Force\_From\_OtherPlanet}}
		}
	\end{algorithm}
	As we see in Algorithm \ref{algonetforce}, it takes \texttt{System} as input, in which there exists information about the total number of planets we are dealing with (\texttt{System.TotalPlanets}) and a ordered list of planets (\texttt{System.AllPlanets}). \texttt{Force\_From\_OtherPlanet} is naturally the gravitational force exerted from one planet onto another as in equation \ref{gravitation} at the position the planet is currently in. 
	
	\subsection{The Euler-Forward Algorithm}
	The Euler-Forward algorithm may be given as a set of discrete recursive relations as
	
	\begin{equation}
		\begin{cases}
		\mathbf{r}_{k+1} = \mathbf{r}_k + h\mathbf{v}_k\\
		\mathbf{v}_{k+1} = \mathbf{v}_k + (h/M_i)\mathbf{F}^{(\text{NET})}_i\\
		\mathbf{r}_0 = (x_0, y_0,z_0),\ \ \mathbf{v}_0 = (v^{(x)}_0, v^{(y)}_0,v^{(z)}_0)
		\end{cases}
		\label{eulerforward}
	\end{equation}
	Counting the number of FLOPs to 5$N$, we may observe that this implementation comes at low computational cost. One could compute the acceleration outside of the algorithm to reduce the number of FLOPs to 4$N$. This however only applies to the iterative recursion, not the program itself! In reality, one should also count the FLOPs involved in calculating the gravitational force between planets to get a true value of the number of FLOPs for the algorithm in its entirety. The FLOPs we present are however representative of the order of the number of FLOPs, and are really only used for comparison between the different algorithms. The Euler - Forward method also carries a global truncation error $\mathcal{O}(h)$, ensuring a linear relationship between the numerical error and the step size $h$. It should also be noted that the Euler-Forward method is \textit{non-symplectic}, meaning that it does not belong to the class of energy-conserving integrators. We shall see the consequence of this in the Results - section. The algorithm may be implemented as\\
	
	
	\begin{algorithm}[H]
		\caption{Euler-Forward Algorithm}
		\label{algoEF}
		\SetAlgoLined
		
		\SetKwData{Left}{left}\SetKwData{This}{this}\SetKwData{Up}{up}
		\SetKwFunction{Union}{Union}\SetKwFunction{FindCompress}{FindCompress}
		\SetKwInOut{Input}{input}\SetKwInOut{Output}{output}
		\Input{System}
		\Output{Current planet position at time $t_i$ ($\mathbf{r} \in \mathbb{R}^3)$}
		\textbf{Var}:\\
		h $\in \mathbb{R}$\\
		\textbf{top}:	\text{current\_planet.Force $\to$ Calculate\_Net\_Force()}
		\text{\textbf{update} curret\_planet.position $\to$ h$\cdot$current\_planet.velocity}
		\text{\textbf{update} current\_planet.velocity $\to$ h$\cdot$current\_planet.Force/current\_planet.mass}
		\newline
	\textbf{goto}: \textbf{top}
	\end{algorithm}
	
	
	\subsection{The Euler - Cromer Algorithm}
	The Euler - Cromer -, or semi-implicit Euler method is a first order symplectic algorithm, meaning that it commits a global truncation error of $\mathcal{O}(h)$. Its symplecticity is said to \textit{almost} conserve physical quantities in conservative fields, making it a suitable candidate for solving the problem at hand. This algorithm may be implemented as a set of discrete recursive relations by

		\begin{equation}
	\begin{cases}
	\mathbf{v}_{k+1} = \mathbf{v}_k + (h/M_i)\mathbf{F}^{(\text{NET})}_i\\
	\mathbf{r}_{k+1} = \mathbf{r}_k + h\mathbf{v}_{k+1}\\
	\mathbf{r}_0 = (x_0, y_0,z_0),\ \ \mathbf{v}_0 = (v^{(x)}_0, v^{(y)}_0,v^{(z)}_0)
	\end{cases}
	\label{eulercromer}
	\end{equation}
	
	The number of FLOPs required to performed $N$ iterations with the Euler-Cromer scheme is, as for the Euler - Forward scheme 5$N$, making it a low cost algorithm, too. The schemes are remarkably similar, as the major difference is that in the Euler Cromer scheme, the position vector is overwritten using the velocity vector one time step ahead, whilst in the Euler Forward scheme it is overwritten using the current velocity vector. This algorithm is implemented as\\
	
	\begin{algorithm}[H]
		\caption{Euler-Cromer Algorithm}
		\label{algoEC}
		\SetAlgoLined
		
		\SetKwData{Left}{left}\SetKwData{This}{this}\SetKwData{Up}{up}
		\SetKwFunction{Union}{Union}\SetKwFunction{FindCompress}{FindCompress}
		\SetKwInOut{Input}{input}\SetKwInOut{Output}{output}
		\Input{System}
		\Output{Current planet position at time $t_i$ ($\mathbf{r} \in \mathbb{R}^3)$}
		\textbf{Var}:\\
		h $\in \mathbb{R}$\\
		\textbf{top}:	\text{current\_planet.Force $\to$ Calculate\_Net\_Force()}
		\text{\textbf{update} current\_planet.velocity $\to$ h$\cdot$current\_planet.Force/current\_planet.mass}
		\text{\textbf{update} curret\_planet.position $\to$ h$\cdot$current\_planet.velocity}
		\newline
	\textbf{goto}: \textbf{top}
	\end{algorithm}
	
	\subsection{The Velocity Verlet Algorithm}
	
	Now, onto the Velocity Verlet algorithm. This algorithm belongs to a whole class of Verlet integrators, and has the initial advantage over the Euler schemes of being a second order symplectic integrator. This means that, in contrasts to the Euler schemes, the Verlet algorithm commits a global truncation error of $\mathcal{O}(h^2)$ such that a ten-fold increase in the temporal resolution leads to a hundred-fold increase in the numerical precision. Simultaneously, its symplecticity ensures conservation of physical quantities in conservative fields. Adding to the list of why the Verlet schemes are interesting, they allow for time-reversibility as the first and final step of the scheme are exactly the same. This means that the integrator could potentially be extended to Hamiltonian systems in which negative time exists! However, departing from the small side track, the Velocity Verlet algorithm may be implemented as a set of discrete recursive relations as
	
		\begin{equation}
	\begin{cases}
	
	\mathbf{r}_{k+1} = \mathbf{r}_k + h\mathbf{v}_{k} + (h^2/2)\mathbf{a}_k\\
	\mathbf{v}_{k+1} = \mathbf{v}_k + (h/2)\left(\mathbf{a}_{k+1} + \mathbf{a}_k\right)\\
	\mathbf{r}_0 = (x_0, y_0,z_0),\ \ \mathbf{v}_0 = (v^{(x)}_0, v^{(y)}_0,v^{(z)}_0)\\
	\mathbf{a}_0 = -\sum_{j=1}^{N}GM_j(\mathbf{r}^{(0)}_{ij}/|\mathbf{r}^{(0)}_{ij}|^3)
	\end{cases}
	\label{vv}
	\end{equation}
	
	Here, $\mathbf{r}^{(0)}_{ij}$ should be read as the radial vector pointing from body $i$ to body $j$ when body $i$ is in its initial position $\mathbf{r}_0 = (x_0,y_0,z_0)$. Counting the number of FLOPs, we see that this algorithm commits a total number of 9$N$ FLOPs. Comparing this against the number of FLOPs committed by the Euler schemes, we see that the Velocity Verlet method is slightly less efficient. The following psuedocode gives an idea of how this is implemented on a computer\\
	
		\begin{algorithm}[H]
		\caption{Velocity Verlet Algorithm}
		\label{algoVV}
		\SetAlgoLined
		
		\SetKwData{Left}{left}\SetKwData{This}{this}\SetKwData{Up}{up}
		\SetKwFunction{Union}{Union}\SetKwFunction{FindCompress}{FindCompress}
		\SetKwInOut{Input}{input}\SetKwInOut{Output}{output}
		\Input{System}
		\Output{Current planet position at time $t_i$ ($\mathbf{r} \in \mathbb{R}^3)$}
		\textbf{Var}:\\
		h $\in \mathbb{R}$\\
		\textbf{top}:	\text{current\_planet.Force $\to$ Calculate\_Net\_Force()}
		\text{\textbf{update} current\_planet.position $\to$ h$\cdot$current\_planet.velocity + $(h^2/2)$current\_planet.Acceleration}
		\text{\textbf{set} curret\_planet.Force\_New $\to$ Calculate\_Net\_Force.at(current\_planet.position)}
		\text{\textbf{update} current\_planet.velocity $\to$ $h/2$(current\_planet.Acceleration\_New + current\_planet.Acceleration)}
		\newline
	\textbf{goto}: \textbf{top}
	\end{algorithm}
	
	\subsection{Unit Tests and Logical Test Functions}\label{tests}
	Unit tests are an essential part of program development as they ensure each part of the implementation produces the results it is supposed to. For this project, we have implemented many such tests, but none of these are worth paying attention to in this article - consult \texttt{unittests.cpp} under the Code-folder in our GITHUB - repository to see these tests. \\
	
	It is worthwhile however, to briefly explain how and why we have implemented logical test functions in our program. In the Results - section we exhibit data about the numerical errors associated with each of the three solvers we study in this project, and a natural question should be how this error computation was performed. Consider the simple Earth-Sun system in which we assume the Earth's orbit is circular. We study this system with good reason, as it allows us not only to compare run times for large $N$ within a suitable time frame, but also because it allows us to design a logical operation for determining the maximum relative error in order to compare numerical precision. Given a circular orbit, the length of Earth's position vector should for each time step have a magnitude of one. Using this, our program contains a test function \texttt{TEST\_circular()} which compares the length of each position vector and the theoretical length. The absolute value of this difference is what we take to be a numerical error. The program then goes on to find the maximum error and returns this.\\
	
	Furthermore, our program incorporates Kepler's second law to determine whether the angular momentum of a planet's orbit is conserved. The magnitude of the position vector relative to the Sun is stored for the previous and current time step, along with the distance travelled along the orbit for the time step. The area of the triangle these vectors make is stored for each time step, and compared to the area of the previous triangle in the function \texttt{TEST\_kepler()}. The areas are compared, and if they differ by more than a set tolerance\footnote{In our case, we have set a tolerance of $10^{-10}$}, the angular momentum for the system is not conserved.
	\section{Results}

\begin{table}[H]
\centering
\caption{Run times for MC method and RK4 method with 10$^5$ integration points and 30 stochastic simulations}
\begin{tabular}{|l|l|l|l|}
\hline
Simulation Time & Average MC [s] & All MC [s]& RK4 [s]\\
\hline
10 & 0.0998 & 2.996 & 1.721 \\
100 & 0.978 & 29.351 & 1.779\\
1000 & 10.182 & 305.479 & 1.723\\
\hline
\end{tabular}
\label{tab:Run_times}
\end{table}

Computational execution times for the program is presented in Table \ref{tab:Run_times}. "Average MC" shows the average time for each Monte Carlo simulation across 30 simulations, while "All MC" pertains to the total time for all 100 simulations. As can plainly be seen from the table, while the computational time for the average Monte Carlo simulation increases linearly, the Runge-Kutta solver is more or less constant for admittedly pretty short simulations. Simulations with times beyond 1000 are unnecessary to show here, as the differences between Monte Carlo and Runge-Kutta are obvious. 

%Plots for kjøring uten vital, og fractions
 \begin{figure}[H]
		\centering
		\begin{subfigure}{0.49\linewidth}
			\includegraphics[width=1.1\linewidth]{Figures/OppgA_4_1_05.png}
			\caption{$\beta = 1$}
		\end{subfigure}
		\begin{subfigure}{0.49\linewidth}
			\includegraphics[width=1.1\linewidth]{Figures/OppgA_4_2_05.png}
			\caption{$\beta = 2$}
		\end{subfigure}
		\begin{subfigure}{0.49\linewidth}
			\includegraphics[width=1.1\linewidth]{Figures/OppgA_4_3_05.png}
			\caption{$\beta = 3$}
		\end{subfigure}
		\begin{subfigure}{0.49\linewidth}
		    \includegraphics[width=1.1\linewidth]{Figures/OppgA_4_4_05.png}
			\caption{$\beta = 4$}
		\end{subfigure}
		\caption{Temporal development of the S, I and R populations, with different initial value for $\beta$, the rate of recovery. $\alpha = 4$ and $\gamma = 0.5$}
		\label{fig:1}
	\end{figure}
	
The four plots in Figure \ref{fig:1} shows the development of the populations of S, I and R with time with different values of $\beta$ for each plot. The values for $\alpha$ and $\gamma$ are held constant. For each plot, the Monte Carlo simulation has been run 100 times. These are the fine red, green and blue lines with varying degree of hue seen in the plots. This allows a deeper shade of color to be interpreted as statistically more common outcomes. The dashed lines show the expectation values at each point in time between all the simulations. The solid lines correspond to the Runge-Kutta solutions.

\begin{table}[H]
\centering	
\begin{tabular}{|c||c|c|c|c|}
\hline
\multicolumn{5}{|c|}{S}\\
\hline
$\beta$&1&2&3&4\\
\hline\hline
$S_\infty$& 100 & 200 & 300 & 400\\
\hline
RK4& 99.75 & 199.90 & 300.24& 397.80\\
\hline
MC& 101.94 & 207.51 & 380.68 & 399.98\\
\hline
\end{tabular}
\begin{tabular}{|c||c|c|c|c|}
\hline
\multicolumn{5}{|c|}{I}\\
\hline
$\beta$&1&2&3&4\\
\hline\hline
$I_\infty$& 100 & 40 & 14.28 & 0 \\
\hline
RK4& 100.08 & 40.07 & 14.06 & 0.21\\
\hline
MC& 98.21 & 37.22 & 1.98 & 0.0\\
\hline
\end{tabular}
\begin{tabular}{|c||c|c|c|c|}
\hline
\multicolumn{5}{|c|}{R}\\
\hline
$\beta$&1&2&3&4\\
\hline\hline
$R_\infty$& 200 & 160 & 85.71 & 0\\
\hline
RK4& 200.17 & 160.03 & 85.70 & 1.99\\
\hline
MC& 199.85 & 155.27 & 17.34 & 0.02\\
\hline
\end{tabular}
\caption{Expectation values for Monte Carlo simulations after equilibration. The final corresponding Runge-Kutta value, and the fraction of people in each population at equilibrium, for each value of $\beta$.}
\label{table:2}
\end{table}

Table \ref{table:2} shows the total number of agents occupying each compartment at equilibrium for each of the four different networks as computed by the expressions in \eqref{sirseq}. This is compared to the final value from the plots in Figure \ref{fig:1}, where "RK4" is the value obtained from the Runge-Kutta solver, and "MC" is the expectation value acquired across all 100 simulations. 



\begin{table}[H]
\centering
\begin{tabular}{|c||l|l||l|l||l|l|}
\hline
\multirow{2}{*}{$\beta$}
    & \multicolumn{2}{c||}{S}
        & \multicolumn{2}{|c||}{I}
            & \multicolumn{2}{|c|}{R} \\   \cline{2-7}
 & $\sigma$ & $\bar{x}$ & $\sigma$ & $\bar{x}$ & $\sigma$ & $\bar{x}$\\ \hline
 1&11.62&101.94&12.39&98.21&9.30&199.85 \\ \hline
 2&29.70&207.51&13.02&37.22&1.99&155.27 \\ \hline
 3&23.96&380.68&3.22&1.98&21.29&17.34 \\ \hline
 4&0.14&399.98&0.0&0.0&0.14&0.02 \\ \hline
\end{tabular}
\caption{Expectation values $\bar{x}$ and standard deviations $\sigma$ for the Monte Carlo simulations after equilibrium.}
\label{table:3}
\end{table}
Table \ref{table:3} shows the expectation values across 100 Monte Carlo simulations at equilibrium as well as the standard deviation, for varying values of $\beta$. 

%VITAL DYNAMICS
 \begin{figure}[H]
		\centering
		\begin{subfigure}{0.49\linewidth}
			\includegraphics[width=1.1\linewidth]{Figures/OppgB_1_0_1.png}
			\caption{$\alpha = 4$, $\beta = 1$, $\gamma=0.5$, $\mu=1$, $\mu^*=0$, $\epsilon=1$}
		\end{subfigure}
		\begin{subfigure}{0.49\linewidth}
			\includegraphics[width=1.1\linewidth]{Figures/OppgB_1_1_1.png}
			\caption{$\alpha = 4$, $\beta = 1$, $\gamma=0.5$, $\mu=1$, $\mu^*=1$, $\epsilon=1$}
		\end{subfigure}
		\begin{subfigure}{0.49\linewidth}
			\includegraphics[width=1.1\linewidth]{Figures/OppgB_1_1_12.png}
			\caption{$\alpha = 4$, $\beta = 1$, $\gamma=0.5$, $\mu=1$, $\mu^*=1$, $\epsilon=1.2$}
		\end{subfigure}
		\begin{subfigure}{0.49\linewidth}
		    \includegraphics[width=1.1\linewidth]{Figures/OppgB_1_2_12.png}
			\caption{$\alpha = 4$, $\beta = 1$, $\gamma=0.5$, $\mu=1$, $\mu^*=2$, $\epsilon=1.2$}
		\end{subfigure}
		\caption{Temporal development of the S, I and R populations, with vital dynamics modelled.}
		\label{fig:vitaldynamics}
	\end{figure}
	
Figure \ref{fig:vitaldynamics} have the SIR-models with vital dynamics enabled. The variables $\mu$, $\mu^*$ and $\epsilon$ govern the conditions respectively for death from natural causes, death from infection, and the birth rate.

%SEASONAL VARIATION
 \begin{figure}[H]
		\centering
		\begin{subfigure}{0.49\linewidth}
			\includegraphics[width=1.1\linewidth]{Figures/OppgC_4_4.png}
			\caption{$\alpha_0 = 4$, $A = 4$, $\omega = 4$}
		\end{subfigure}
		\begin{subfigure}{0.49\linewidth}
			\includegraphics[width=1.1\linewidth]{Figures/OppgC_4_2.png}
			\caption{$\alpha_0 = 4$, $A = 4$, $\omega = 2$}
		\end{subfigure}
		\begin{subfigure}{0.49\linewidth}
			\includegraphics[width=1.1\linewidth]{Figures/OppgC_8_4.png}
			\caption{$\alpha_0 = 4$, $A = 8$, $\omega = 4$}
		\end{subfigure}
		\begin{subfigure}{0.49\linewidth}
		    \includegraphics[width=1.1\linewidth]{Figures/OppgC_4_025.png}
			\caption{$\alpha_0 = 4$, $A = 4$, $\omega = 0.25$}
		\end{subfigure}
		\caption{Development with seasonal variation enabled. Different values for $A$ and $\omega$ have been used.}
		\label{fig:seasonalvar}
	\end{figure}
Figure \ref{fig:seasonalvar} displays four different simulations of the spread of disease in a network where we enable seasonal variation in the rate of infection, $\alpha(t)$. 
%Vaccination
 \begin{figure}[H]
		\centering
		\begin{subfigure}{0.49\linewidth}
			\includegraphics[width=1.1\linewidth]{Figures/Vax_Pulse2t_ckonst_4_1_05.png}
			\caption{Pulse vaccination, constant $\gamma$}
		\end{subfigure}
		\begin{subfigure}{0.49\linewidth}
			\includegraphics[width=1.1\linewidth]{Figures/Vax_Pulse2t_cvar_4_1_05.png}
			\caption{Pulse vaccination, variable $\gamma$}
		\end{subfigure}
		\begin{subfigure}{0.49\linewidth}
			\includegraphics[width=1.1\linewidth]{Figures/Vax_expt_15x03_ckonst_4_1_05.png}
			\caption{Exponential vaccination, constant $\gamma$}
		\end{subfigure}
		\begin{subfigure}{0.49\linewidth}
		    \includegraphics[width=1.1\linewidth]{Figures/Vax_expt_15x03_cvar_4_1_05.png}
			\caption{Exponential vaccination, variable $\gamma$}
		\end{subfigure}
		\caption{Development with vaccination with $\alpha = 4$, $\beta = 1$, $\gamma=0.5$ after time 10 for a and b and after time 15 for c and d}
		\label{fig:vaccinationvar}
	\end{figure}

Figure \ref{fig:vaccinationvar} displays the results obtained for simulations in which different vaccination programs are initiated after a given time $t$. In Figure \ref{fig:vaccinationvar} (b) we have implemented a function which decreases the rate of recovery $\gamma$ as discussed in Section \ref{section:vacfunc}, and the same was done in Figure \ref{section:vacfunc}. In these plots the ODE solutions are not included as the variable rate of recovery was never implemented for this solver. 


	\section{Discussion}
In this numerical project, we have undertaken the task of gradually imposing layers of realism in the SIRS model. Before changing the model at all, we obtain results which may be seen in Figure \ref{fig:1} and Tables \ref{table:2} and \ref{table:3}. These act at a first glance as a comparison between the deterministic and stochastic approach to simulating the spread of disease in a network. Most notably, the tabulated number of agents occupying each compartment at equilibrium in Table \ref{table:2} allows for a more detailed comparison between the two. We may observe here that for $\beta = 3$, the expected number of agents in $S$ at equilibrium approaches 400, rather than the theoretical value of $300$. Following this deviation, the expected number of infected and recovered agents tend to zero - a very interesting outcome as for a recovery rate that is lower than the infection rate, one would expect the disease to remain within the population at a non-zero equilibrium. For $\beta = 4$, the trend for both the deterministic and stochastic solution is that the total occupation of $I$ tends to zero, i.e. the disease is autonomously eradicated as expected. Concerning the tabulated data, we may also notice that for $\beta = 2,3$ the standard deviation in the data representing the total number of occupants in each compartment at equilibrium increases dramatically compared to the standard deviation in simulations devising $\beta = 1,4$. These large deviations are also easily recognized on the plots where there are a wide range of trajectories undertaken in the different simulations. We believe this has to do with the non-linearity of the system. If we turn our attention to Figure \ref{fig:vitaldynamics}, the large deviations become even more pronounced when adding terms in the non-linear system, leading us to believe that this is the main reason for the increasing standard deviations. Plots of the temporally evolving standard deviation may be found in Appendix B.\\

The results obtained from adding the realization of vital dynamics are displayed in Figure \ref{fig:vitaldynamics}. This simple improvement alone changes the evolution of disease contagion within a network dramatically. In Figure \ref{fig:vitaldynamics} (a), the birth rate and death rate are set to be equal, and the death rate due to infection is held at zero. As there are more possible transitions into the state of having died, than there are possibilities of being born or 'added' into the network, this simulation tends to zero for all three compartments eventually. However, here we wanted to show that for societies with an insignificant, yet slightly negative population growth, the occupation of $I$ equilibrates at a level which is approximately 20\% higher than for the same simulation without vital dynamics within the first 10 time units. Figure \ref{fig:vitaldynamics} (b) simulates a situation in which the infectious disease has a high mortality rate, here set to be equal to both the birth - and death rate. Here we see that in the first time units, the likelihood of dying, whether from natural causes or the disease, is very high. This leads to a rapid fall in the total population. After some time, all agents in the network occupy $S$, and therefore, the addition and subtraction of agents are in equilibrium as $P(B\to S) = P(S\to D)$ such that the system sets at a non - zero value. More importantly, this only occurs in the stochastic simulation, as the ODE solver suggests that the network size tends to zero after approximately 30 time units. In Figures \ref{fig:vitaldynamics} (c) and (d), the same process as for (b) occurs, but now we simulate a network in which the population growth is positive. 
\\ \\
With seasonal variation added to the model, the rate of transmission $\alpha$ varies periodically with respect to time, and the trends in transitions in the network can be seen in Figure \ref{fig:seasonalvar}. Figures (a), (b) and (c) shows the effect of varying amplitudes and frequencies in a way one would expect. After equilibrating, the agents in R dominate, and the oscillations caused by the varying rate of transmission is evident. In Figure \ref{fig:seasonalvar} (d) an interesting equilibrium is reached. The Runge-Kutta solution and the Monte-Carlo expectation values diverge around time step 18. The deterministic solution oscillates periodically with areas of extremely high rates of susceptible agents and very few infected agents, before the contaigon flares up again. For the stochastic simulation, there appears to be two variations. One where it follows the deterministic simulation, and one where the number of agents who are susceptible reaches the network size, and the infection is eradicated. As can also be seen, at the high points of S, the stochastic solution will either follow the path to eradicate the contaigon, or follow the deterministic path and have a relapse in infection. At the next high point, the stochastic solution can make this choice of path once more. This shows in a pretty way the advantage a random sampling stochastic solution has over a deterministic one. In real life, there will always be an element of randomness and uncertainty, because of the impossibility of knowing absolutely all variables associated with change. With the Monte Carlo simulations, we are able to mimic the randomness of reality somewhat closer than we are with the Runge-Kutta simulations.\\

The last plots included in the Results section describe the effect of vaccinating susceptible agents, as to break the cyclic nature of the SIRS model by directly moving them from $S$ to $R$. As we can see from Figure \ref{fig:vaccinationvar} (c) and (d), had there been a possibility of continuously vaccinating the agents in $S$, this would lead to a eradication of the disease within a short period of time. However, as this vaccination program is less than economically feasible in many countries, we may rather turn our attention to Figures \ref{fig:vaccinationvar} (a) and (b) as these describe the implementation of a more realistic pulse vaccination program. In simulation (a) we kept $\gamma$ constant, leading to the issue that vaccinated agents may transition back into $S$. In (b), however, we used the method described in Section \ref{section:vacfunc} in order to better ensure that $R$ - occupants do not leave the compartment when already vaccinated. The vaccination function used in these simulations was

$$
f(t)
\begin{cases}
2t, \ \text{if}\ \cos(t) > 0\\
0, \ \text{else}
\end{cases}
$$

as to simulate a pulse with linearly increasing value. Here we wanted to capture a situation in which susceptible agents are vaccinated on regular intervals, but in each interval the availability of the vaccine increases. A plot of the same simulation where the rate of vaccination is a constant percentage of the current number of occupants of $S$ at each interval is included in Appendix C. The effect of pulse vaccination are clear from the presented plots. In the intervals where susceptible agents are vaccinated, the number of infected agents fall dramatically, and as the likelihood of transitioning back into $S$ becomes smaller after each period of vaccination, the next maximum in the infected data will be smaller than the previous one. As a consequence, the number of infected agents steadily decreases in an oscillatory fashion, until it reaches zero. In Figure \ref{fig:vaccinationvar} (b), at approximately 48 time units, the data represented in a light blue hue indicates that in some simulations the disease flares even after the number of infected agents reaches zero. This is a common feature in pulse vaccination, in which there are effectively disease free periods, right before a 'freak' outbreak which is quickly eradicated. This phenomenon is more so visible in Figure \ref{fig:All} in Appendix A. It should also be noted that, even though the latter vaccination program is more realistic, an important difference emerges when comparing the two. The results for a network in which the exponentially increasing rate of vaccination is implemented tends to a disease free equilibrium much faster than for the network dictated by pulse vaccination, leading up to the fact that higher rates of vaccination has a higher likelihood of eradicating a communicable disease. This stresses the idea of mass participation in vaccination programs in order to achieve so-called herd immunity.  
\\

Up until this point we have presented select results from our studies of the SIRS model. Although these results serve the purpose of highlighting key findings about the effects of additional realizations to our model, we wish to emphasize that the SIRS model is severely flawed by assuming that the network and the agents it contains mixes homogeneously. In reality, the heterogeneous mixing of a network plays an important role in determining how an infectious disease spreads. Some agents may contact the disease more frequently than others, some may by chance never contact the disease within a given time period. On top of this, the SIRS model fails to further propagate the idea of individuality on a more complex level. Especially for simulations of populations containing sentient human beings, a highly realistic model should take rational behaviour into account. Some people have lower propensity towards being vaccinated or having their children vaccinated, some are more likely to sustain a higher level of personal hygiene, and a select few might even choose to commit to self-isolation. A major improvement on the SIRS model would be to devise it in conjunction with a spatial model\footnote{In general a diffusion model is used.}, as well as object orienting our code in order to initialize nodes on a spatial grid with different health status IDs.
	\section{Conclusive Remarks}
In summary, we have made some remarkable findings in this rather large study spanning across multiple topics. Although there are some aspects of the study which failed to produce solid results, it has overall revealed profound insight into numerical differential equation solvers, object orientation and our Solar System. Firstly, we have found that the Velocity Verlet algorithm proves superior to both the Euler - Forward and Euler - Cromer method. Its error displays a linear relationship with the temporal resolution which changes twice as fast as for the two Euler methods, and this comes at the expense of a run time which is slightly longer - yet insignificantly so. We made a good decision when choosing to include the Euler - Cromer method as a part of the study, as this also allowed us to compare two symplectic integrators. From what we have seen, the committed error in the two algorithms dominates over their ability to display symplecticity, and to conduct more precise studies of possible differences one should have a computer (or several) capable of performing simulations for much greater temporal resolutions than we have. Furthermore, we have at this point designed a program which, in light of object orientation, is highly functioning. It seamlessly creates large systems and produces results which behave similarly for different temporal resolutions. Moreover, it is constructed such that it is very easy to create systems of non-planetary objects and study these instead. Concerning the results procured from this study other than those relating to the algorithms and the object orientation of the code, there are next to none. We have obtained results regarding the perihelion precession of Mercury, but the findings are inconclusive.
	
	\newpage
	\addcontentsline{toc}{section}{References}
	\begin{thebibliography}{5}
        \bibitem{statphys}
        M. Plischke and B. Bergersen, \textit{Equilibrium State Physics}, World Scientific, chapters 5 and 6
        \bibitem{nightingale}
        Nightingale, M. P. and Bl\"ote, H. W. J. (1996), \textit{Dynamic Exponent of the Two-Dimensional Ising Model and Monte Carlo Computation of the Subdominant Eigenvalue of the Stochastic Matrix}. Phys. Rev. Lett. 76, 24, pp. 4548-4551
        \bibitem{walter}
        J.-C. Water and G.T. Barkema, \textit{An introduction to Monte Carlo methods}. PDF, 6-8. \\
        \url{https://arxiv.org/pdf/1404.0209.pdf}
        \bibitem{carlon}
        Carlon, E. (2013), \textit{Advanced Monte Carlo Methods.} PDF, 15-16\\
        \url{http://itf.fys.kuleuven.be/~enrico/Teaching/monte_carlo_2012.pdf}
        \bibitem{gould}
        Harvey, G. and Tobochnik, J. (1989), \textit{Overcoming Critical Slowing Down}. PDF, 1-3.\\
        \url{https://aip.scitation.org/doi/pdf/10.1063/1.4822858}
	\end{thebibliography}
	
\end{document} 