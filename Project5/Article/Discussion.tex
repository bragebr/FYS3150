\section{Discussion}
In this numerical project, we have undertaken the task of gradually imposing layers of realism in the SIRS model. Before changing the model at all, we obtain results which may be seen in Figure \ref{fig:1} and Tables \ref{table:2} and \ref{table:3}. These act at a first glance as a comparison between the deterministic and stochastic approach to simulating the spread of disease in a network. Most notably, the tabulated number of agents occupying each compartment at equilibrium in Table \ref{table:2} allows for a more detailed comparison between the two. We may observe here that for $\beta = 3$, the expected number of agents in $S$ at equilibrium approaches 400, rather than the theoretical value of $300$. Following this deviation, the expected number of infected and recovered agents tend to zero - a very interesting outcome as for a recovery rate that is lower than the infection rate, one would expect the disease to remain within the population at a non-zero equilibrium. For $\beta = 4$, the trend for both the deterministic and stochastic solution is that the total occupation of $I$ tends to zero, i.e. the disease is autonomously eradicated as expected. Concerning the tabulated data, we may also notice that for $\beta = 2,3$ the standard deviation in the data representing the total number of occupants in each compartment at equilibrium increases dramatically compared to the standard deviation in simulations devising $\beta = 1,4$. These large deviations are also easily recognized on the plots where there are a wide range of trajectories undertaken in the different simulations. We believe this has to do with the non-linearity of the system. If we turn our attention to Figure \ref{fig:vitaldynamics}, the large deviations become even more pronounced when adding terms in the non-linear system, leading us to believe that this is the main reason for the increasing standard deviations. Plots of the temporally evolving standard deviation may be found in Appendix B.\\

The results obtained from adding the realization of vital dynamics are displayed in Figure \ref{fig:vitaldynamics}. This simple improvement alone changes the evolution of disease contagion within a network dramatically. In Figure \ref{fig:vitaldynamics} (a), the birth rate and death rate are set to be equal, and the death rate due to infection is held at zero. As there are more possible transitions into the state of having died, than there are possibilities of being born or 'added' into the network, this simulation tends to zero for all three compartments eventually. However, here we wanted to show that for societies with an insignificant, yet slightly negative population growth, the occupation of $I$ equilibrates at a level which is approximately 20\% higher than for the same simulation without vital dynamics within the first 10 time units. Figure \ref{fig:vitaldynamics} (b) simulates a situation in which the infectious disease has a high mortality rate, here set to be equal to both the birth - and death rate. Here we see that in the first time units, the likelihood of dying, whether from natural causes or the disease, is very high. This leads to a rapid fall in the total population. After some time, all agents in the network occupy $S$, and therefore, the addition and subtraction of agents are in equilibrium as $P(B\to S) = P(S\to D)$ such that the system sets at a non - zero value. More importantly, this only occurs in the stochastic simulation, as the ODE solver suggests that the network size tends to zero after approximately 30 time units. In Figures \ref{fig:vitaldynamics} (c) and (d), the same process as for (b) occurs, but now we simulate a network in which the population growth is positive. 
\\ \\
With seasonal variation added to the model, the rate of transmission $\alpha$ varies periodically with respect to time, and the trends in transitions in the network can be seen in Figure \ref{fig:seasonalvar}. Figures (a), (b) and (c) shows the effect of varying amplitudes and frequencies in a way one would expect. After equilibrating, the agents in R dominate, and the oscillations caused by the varying rate of transmission is evident. In Figure \ref{fig:seasonalvar} (d) an interesting equilibrium is reached. The Runge-Kutta solution and the Monte-Carlo expectation values diverge around time step 18. The deterministic solution oscillates periodically with areas of extremely high rates of susceptible agents and very few infected agents, before the contaigon flares up again. For the stochastic simulation, there appears to be two variations. One where it follows the deterministic simulation, and one where the number of agents who are susceptible reaches the network size, and the infection is eradicated. As can also be seen, at the high points of S, the stochastic solution will either follow the path to eradicate the contaigon, or follow the deterministic path and have a relapse in infection. At the next high point, the stochastic solution can make this choice of path once more. This shows in a pretty way the advantage a random sampling stochastic solution has over a deterministic one. In real life, there will always be an element of randomness and uncertainty, because of the impossibility of knowing absolutely all variables associated with change. With the Monte Carlo simulations, we are able to mimic the randomness of reality somewhat closer than we are with the Runge-Kutta simulations.\\

The last plots included in the Results section describe the effect of vaccinating susceptible agents, as to break the cyclic nature of the SIRS model by directly moving them from $S$ to $R$. As we can see from Figure \ref{fig:vaccinationvar} (c) and (d), had there been a possibility of continuously vaccinating the agents in $S$, this would lead to a eradication of the disease within a short period of time. However, as this vaccination program is less than economically feasible in many countries, we may rather turn our attention to Figures \ref{fig:vaccinationvar} (a) and (b) as these describe the implementation of a more realistic pulse vaccination program. In simulation (a) we kept $\gamma$ constant, leading to the issue that vaccinated agents may transition back into $S$. In (b), however, we used the method described in Section \ref{section:vacfunc} in order to better ensure that $R$ - occupants do not leave the compartment when already vaccinated. The vaccination function used in these simulations was

$$
f(t)
\begin{cases}
2t, \ \text{if}\ \cos(t) > 0\\
0, \ \text{else}
\end{cases}
$$

as to simulate a pulse with linearly increasing value. Here we wanted to capture a situation in which susceptible agents are vaccinated on regular intervals, but in each interval the availability of the vaccine increases. A plot of the same simulation where the rate of vaccination is a constant percentage of the current number of occupants of $S$ at each interval is included in Appendix C. The effect of pulse vaccination are clear from the presented plots. In the intervals where susceptible agents are vaccinated, the number of infected agents fall dramatically, and as the likelihood of transitioning back into $S$ becomes smaller after each period of vaccination, the next maximum in the infected data will be smaller than the previous one. As a consequence, the number of infected agents steadily decreases in an oscillatory fashion, until it reaches zero. In Figure \ref{fig:vaccinationvar} (b), at approximately 48 time units, the data represented in a light blue hue indicates that in some simulations the disease flares even after the number of infected agents reaches zero. This is a common feature in pulse vaccination, in which there are effectively disease free periods, right before a 'freak' outbreak which is quickly eradicated. This phenomenon is more so visible in Figure \ref{fig:All} in Appendix A. It should also be noted that, even though the latter vaccination program is more realistic, an important difference emerges when comparing the two. The results for a network in which the exponentially increasing rate of vaccination is implemented tends to a disease free equilibrium much faster than for the network dictated by pulse vaccination, leading up to the fact that higher rates of vaccination has a higher likelihood of eradicating a communicable disease. This stresses the idea of mass participation in vaccination programs in order to achieve so-called herd immunity.  
\\

Up until this point we have presented select results from our studies of the SIRS model. Although these results serve the purpose of highlighting key findings about the effects of additional realizations to our model, we wish to emphasize that the SIRS model is severely flawed by assuming that the network and the agents it contains mixes homogeneously. In reality, the heterogeneous mixing of a network plays an important role in determining how an infectious disease spreads. Some agents may contact the disease more frequently than others, some may by chance never contact the disease within a given time period. On top of this, the SIRS model fails to further propagate the idea of individuality on a more complex level. Especially for simulations of populations containing sentient human beings, a highly realistic model should take rational behaviour into account. Some people have lower propensity towards being vaccinated or having their children vaccinated, some are more likely to sustain a higher level of personal hygiene, and a select few might even choose to commit to self-isolation. A major improvement on the SIRS model would be to devise it in conjunction with a spatial model\footnote{In general a diffusion model is used.}, as well as object orienting our code in order to initialize nodes on a spatial grid with different health status IDs.