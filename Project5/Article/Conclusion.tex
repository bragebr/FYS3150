\section{Conclusive Remarks}
The SIRS model is an excellent choice for heuristically studying how an infectious disease spreads within a network on the temporal dimension alone. From our select results, it has become clear that both the deterministic and stochastic model approaches to simulating such networks are in every way valid, although the stochastic approach produce results which capture the random nature of such processes in a more organic way, and that are statistically more realistic. It predicts outcomes which deviate 
from what may be produced by solving a set of differential equations deterministically, which in turn may be analysed in detail. We have added different realizations in our model, such as seasonal variation in the rate of transmission, vital dynamics and vaccination programs, finally arriving at a compound model for even more realistic simulations. Upon the addition of new layers, we are able to get new insight into the essential mechanics of disease spread. Most notably we find that for diseases with high mortality rates, the compartment containing susceptible agents eventually equilibrates at a non-zero level, whilst the two other compartments reaches equilibria containing no agents. Furthermore, in the case of vaccination, we find that vaccinating the agents at a higher rate leads to a much faster tendency towards zero in the infected compartment, which is what we would expect. As to what concerns the performance of our program, we have made some summary remarks on the inherent flaws in the SIRS model we have devised. We have concluded that conjoining the temporal SIRS model with a spatial diffusion model (as an example) will allow for more realistic simulations of the process of disease spread. Object orienting our code will also be helpful in tracking the health status of all of the initialized agents, and is a task that lies ahead of us in our academic career. 