\section{Introduction}
Just short of one year ago, the world first learnt of the Covid-19 virus, which has since then gone on to change the everyday life of people all around the globe. The family of coronaviruses (CoV) have been identified many times before, causing illnesses which may manifest themselves somewhere between the common cold and more severe disease. The introduction of a novel virus, however, calls for detailed studies of how different social and medical measures, alongside seasonal variation and vital dynamics have an impact on the short - and long term development of the number of infected agents within a network. The wording we devise in this article is deliberate, as this model does not only apply to human beings in a larger population, but also to other entities such as biological cells, or other non-sentient nodes in networks in which we may, as an example, study the communicable flow of information.
This study is designed to develop a numerical model for the spread of contagious disease in a network of agents. To do this, we work via the SIRS\footnote{Susceptible - Infected - Recovered - Susceptible} model and implement it numerically first through a set of coupled ordinary differential equations (ODEs), and later by a stochastic model in order to also capture the random nature of reality in our simulations. The stochastic modelling is in itself an interesting study, as the model may be applied in different areas of research such as information, rumor and substance addiction dynamics in a network. We will first go into detail on the SIR(S) model and its underlying assumptions, before we move on to explain how it may be numerically implemented, whether it is deterministically or stochastically. We shall then present select findings from our studies, and discuss them in light of both the numerical implementation and the effects of adding several layers of realism such as seasonal variation in rate of contagion and vaccination measures. Lastly, we make some summary remarks on the presented work.
\newpage