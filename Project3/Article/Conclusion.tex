\section{Conclusive Remarks}
In summary, we have made some remarkable findings in this rather large study spanning across multiple topics. Although there are some aspects of the study which failed to produce solid results, it has overall revealed profound insight into numerical differential equation solvers, object orientation and our Solar System. Firstly, we have found that the Velocity Verlet algorithm proves superior to both the Euler - Forward and Euler - Cromer method. Its error displays a linear relationship with the temporal resolution which changes twice as fast as for the two Euler methods, and this comes at the expense of a run time which is slightly longer - yet insignificantly so. We made a good decision when choosing to include the Euler - Cromer method as a part of the study, as this also allowed us to compare two symplectic integrators. From what we have seen, the committed error in the two algorithms dominates over their ability to display symplecticity, and to conduct more precise studies of possible differences one should have a computer (or several) capable of performing simulations for much greater temporal resolutions than we have. Furthermore, we have at this point designed a program which, in light of object orientation, is highly functioning. It seamlessly creates large systems and produces results which behave similarly for different temporal resolutions. Moreover, it is constructed such that it is very easy to create systems of non-planetary objects and study these instead. Concerning the results procured from this study other than those relating to the algorithms and the object orientation of the code, there are next to none. We have obtained results regarding the perihelion precession of Mercury, but the findings are inconclusive.